\documentclass{article}
\usepackage[utf8]{inputenc}
\usepackage[spanish]{babel}
\usepackage{graphicx, graphics, float, hyperref}
\usepackage{listings}
\usepackage[a4paper, total={6in, 10in}]{geometry}

\title{DSD Memoria Práctica 4}
\author{Andrés Merlo Trujillo}
\date{}
\hypersetup{
    colorlinks=true,
    linkcolor=black,
}

\begin{document}
%TODO: COMPROBAR EN LA PARTE DE SOCKET IO QUE PARTES SON EXACTAMENTE SOCKETS.
%TODO 2: MEJORAR LA EXPLICACIONES EN GENERAL, SOBRE TODO LA PARTE DE LOS SOCKETS QUE ESTA MUY RARA.

\maketitle

\tableofcontents

\newpage

\section{Introducción}
En esta práctica se pide probar unos ejemplos de prueba y comentarlos, e implementar un sistema de domótica básico.

Para ello, es necesario tener instalado Node.js y MongoDB. También es necesario instalar NPM para poder instalar los módulos de Socket.io y el de MongoDB (driver para la conexión con la BBDD).

\bigskip

Un problema que he tenido ha sido que no podía ejecutar el archivo de prueba \textit{mongo-test.js}. Mostraba el mensaje de que el servidor se había iniciado, pero al intentar acceder con la URL ponía que la conexión había sido rechazada. Tras pasar un periodo de tiempo de unos 40 segundos, el programa mostraba un mensaje de error.

Tras depurar el programa en busca del mensaje de error, me di cuenta de que localhost se resolvía a \textit{::1}, que es la versión IPv6. Además al hacer ping a localhost obtenía lo siguiente:

\begin{figure}[H]
    \centering
    \includegraphics[width=0.7\textwidth]{images/ping.png}
    \caption{Salida que salía. Al estar arreglado sale localhost6, pero ponía localhost antes.}
\end{figure}

Por tanto, siempre se estaba resolviendo a la dirección IPv6.

\bigskip

MongoDB por defecto no usa IPv6, por lo que hay dos opciones para que funcione:

\begin{itemize}
    \item Permitir a MongoDB que use también las direcciones IPv6 o.
    
    \item Modificar el archivo \textit{/etc/hosts} para que use otro nombre distinto.
\end{itemize}

Al final opté por la segunda opción. Para ello llamé a ``localhost'' en IPv6 como ``localhost6''. Ahora al hacer ping sí se obtiene la dirección loopback de IPv4:

\begin{figure}[H]
    \centering
    \includegraphics[width=0.7\textwidth]{images/pingOk.png}
    \caption{Salida después del arreglo.}
\end{figure}

%QUIZAS SI DA MAS FALLOS MONGO PONERLOS

\newpage

\section{Primera parte. Ejemplos}
\subsection{helloworld.js}
\subsubsection{Descripción}
Este ejemplo lo único que realiza es mostrar por pantalla en el navegador el mensaje ``Hola mundo''. Adicionalmente, también muestra por la terminal del servidor el JSON con toda la información de la petición del cliente (navegador).

\begin{figure}[H]
    \centering
    \begin{minipage}[H]{0.49\textwidth}
        \centering
        \includegraphics[width=\textwidth]{images/helloworldC.png}
        \caption{Salida del cliente.}
    \end{minipage}
    \hfill
    \begin{minipage}[H]{0.49\textwidth}
        \centering
        \includegraphics[width=\textwidth]{images/helloworldS.png}
        \caption{Salida del servidor cuando un cliente se conecta.}
    \end{minipage}
\end{figure}

\subsubsection{Implementación}
Para implementarlo, lo primero que hay que hacer es montar el servidor HTTP. Para ello, se importa el módulo de Node.js ``http''.

Una vez importado, se crea el servidor con el método ``createServer'' cuyo argumento es una función que se ejecuta cada vez que se realiza una petición HTTP.

Dentro del cuerpo de esta función, para que el servidor muestre la información de la petición, es necesario usar ``request.headers'' y pasarlo al log de la consola de JavaScript.

Para que el cliente reciba el mensaje, es necesario escribir una cabecera básica con el código 200, indicando que es correcto y el tipo MIME, que en este caso será texto plano.

Luego con el método write, se escribe el mensaje ``Hola mundo'' en el cliente. Por último, se finaliza con ``end'' para que se mande el mensaje.

\begin{figure}[H]
    \centering
    \includegraphics[width=0.6\textwidth]{images/header.png}
    \caption{Cabecera de la petición realizada en Firefox.}
\end{figure}

\subsection{calculadora.js}
\subsubsection{Descripción}
Este ejemplo implementa una calculadora básica haciendo uso de Node.js. 

\bigskip

A la calculadora se le pasa el operador y los operandos en la URL en el navegador. En caso de ser incorrectos o no tener suficientes, muestra en el navegador un error. Las operaciones que puede hacer son: sumar, restar, producto y dividir.

\bigskip

La sintaxis de la URL es la siguiente: 

\verb|http://localhost:8080/<operador>/<operando-1>/<operando-2>|

\bigskip

\begin{figure}[H]
    \centering
    \includegraphics[width=0.5\textwidth]{images/rescalcnormal.png}
    \caption{Resultado de sumar 1 y 2 en Chrome.}
\end{figure}

\subsubsection{Implementación}
La parte de creación del servidor es exactamente igual que en la que se realiza en ``helloworld.js''. Se ha creado una función ``calcular'' que permite realizar las funciones de la calculadora.

\bigskip

También se usa el módulo ``url'' para obtener la URL que pasa el cliente. El objetivo es conseguir obtener los parámetros para pasarlos a la función y que realice la cuenta. Para ello, se usa ``url.parse'' con el ``request.url'' para obtener más información de la URL y luego solo se escoge el campo ``pathname''. Esto hace que se obtenga la URI, por ejemplo, si se quiere hacer ``1+2'': ``/sumar/1/2''.

Cabe notar que, en este caso, la línea que hace esto es innecesaria, ya que se puede hacer un simple ``request.url'' y obtener la URI deseada.

\bigskip

Luego, se elimina el primer ``/'' para poder hacer luego split con el mismo delimitador para obtener un array con el operador el primero, y los operadores al final. A continuación, se pasa a float y se llama a la función ``calcular''.

\bigskip

Por último, se manda de manera idéntica a la explicada en la sección anterior para el ``helloworld.js''.

\begin{figure}[H]
    \centering
    \includegraphics[width=0.5\textwidth]{images/dividircnormal.png}
    \caption{Resultado de dividir 1 entre 2 en Chrome.}
\end{figure}

\newpage

\subsection{calculadora-web.js}
\subsubsection{Descripción}
Este ejemplo implementa exactamente lo mismo que ``calculadora.js'', pero teniendo una interfaz web que simplifica la introducción de los valores.

También hace uso de Node.js y de los módulos adicionales ``path'' y ``fs'' para buscar el archivo HTML con la ruta absoluta y también para comprobar la existencia de archivos en el sistema, y de leerlos para ofrecerlos en el cliente.

\begin{figure}[H]
    \centering
    \begin{minipage}[H]{0.49\textwidth}
        \centering
        \includegraphics[width=\textwidth]{images/cwebsuma.png}
        \caption{Resultado de sumar 1 y 2 en Chrome.}
    \end{minipage}
    \hfill
    \begin{minipage}[H]{0.49\textwidth}
        \centering
        \includegraphics[width=\textwidth]{images/cwebsumaserver.png}
        \caption{Salida del servidor.}
    \end{minipage}
\end{figure}


Además muestra por la pantalla donde se ejecuta la petición REST que se ha realizado, que sigue la misma sintaxis que el ejemplo anterior.

\subsubsection{Implementación}
La implementación es muy similar al anterior, pero en este caso se usa el módulo de Node.js ``fs'' para poder realizar operaciones con archivos.

\bigskip

Lo primero que tiene que hacer es enviar la página al navegador. Para ello, se consigue la URI como antes, luego se comprueba que si el usuario solo ha puesto la dirección (en este caso localhost:8080), en caso afirmativo se pone que sirva el archivo HTML. A continuación, se le añade la ruta absoluta de ese archivo y, finalmente, con el método ``exists'' comprueba que existe, y en caso afirmativo, con el método ``readFile'' se carga el archivo HTML de una manera similar a como se hacía para los mensajes en los ejemplos anteriores.

\bigskip

Una vez cargada la página, el usuario puede insertar los operandos y el operador y, al darle a enviar se ejecuta la función ``enviar'' que lo que hace es convertir los valores introducidos en una cadena del tipo ``operador/operando1/operando2''. Luego, realiza una petición HTTP GET al servidor con esta cadena, la cual al pasar por el método ``exists'' va a devolver que no existe porque no es un archivo y va a ejecutar el mismo código que en el ejemplo de la calculadora anterior, devolviendo así el resultado de la operación.

\begin{figure}[H]
    \centering
    \begin{minipage}[H]{0.49\textwidth}
        \centering
        \includegraphics[width=\textwidth]{images/cwebprod.png}
        \caption{Resultado de multiplicar 3 por 8 en Chrome.}
    \end{minipage}
    \hfill
    \begin{minipage}[H]{0.49\textwidth}
        \centering
        \includegraphics[width=\textwidth]{images/cwebprodserver.png}
        \caption{Salida del servidor.}
    \end{minipage}
\end{figure}

\subsection{connections.js}
\subsubsection{Descripción}
Este ejemplo hace uso del módulo Socket.io para poder actualizar todos los clientes de manera asíncrona y realizar una comunicación entre el cliente y el servidor para que todos tengan la información más actualizada.

Cuando un cliente se conecta, este aparece en la pantalla del navegador con su dirección IP y el puerto usado. Además, si se abren más pestañas se puede ver como aparece la dirección y puerto del nuevo cliente, junto a los que ya estaban conectados. Si nos dirigimos de nuevo a la pestaña del primer cliente se puede ver que se actualizan las conexiones también. Esto mismo pasa también para las desconexiones, cuando un cliente se desconecta los demás lo ven en su sesión.

En la pantalla del servidor aparecen los clientes conectados, las nuevas conexiones y cuando un usuario se va a desconectar.

\begin{figure}[H]
    \centering
    \begin{minipage}[H]{0.49\textwidth}
        \centering
        \includegraphics[width=\textwidth]{images/connect1.png}
        \caption{Un solo cliente conectado.}
    \end{minipage}
    \hfill
    \begin{minipage}[H]{0.49\textwidth}
        \centering
        \includegraphics[width=\textwidth]{images/connect2.png}
        \caption{Dos clientes conectados}
    \end{minipage}
\end{figure}

Además, cuando el cliente se desconecta del servidor, aparece un mensaje informándolo.

\begin{figure}[H]
    \centering
    \begin{minipage}[H]{0.49\textwidth}
        \centering
        \includegraphics[width=\textwidth]{images/connectserver.png}
        \caption{Salida del servidor cuando se conectan dos.}
    \end{minipage}
    \hfill
    \begin{minipage}[H]{0.49\textwidth}
        \centering
        \includegraphics[width=\textwidth]{images/disconnectserver.png}
        \caption{Salida del servidor cuando un cliente se desconecta.}
    \end{minipage}
\end{figure}

\subsubsection{Implementación}
Para implementar este ejemplo es necesario crear un ``httpServer'' como se ha hecho en los ejemplos anteriores, pero esta vez solo cargando el fichero HTML.

\bigskip

También es necesario crear el objeto Socket.io e indicarle la funcionalidad y los eventos que debe escuchar y emitir en cuanto se conecte un cliente. En este caso, cuando un cliente se conecta, lo que se hace primero es añadir el usuario al array, luego se emite a todos los suscritos el cambio y por último se configuran los eventos de escucha ``output-evt'' y ``disconnect'' para mostrar un mensaje por el navegador y para actualizar los clientes conectados respectivamente, cuando el cliente emita alguno de estos eventos.

En el fichero HTML se conecta al socket del servidor y se configuran los eventos que va a suscribirse. El evento ``connect'' se ejecuta en cuanto se conecta y le manda un evento del tipo ``output-evt'' al servidor, el cual devuelve de vuelta ``Hola Cliente!'' y lo capta el cliente en ``output-evt'', que actualiza el mensaje de la página HTML.

Cuando un cliente se desconecta se ejecuta en el servidor el evento ``disconnect'', el cual elimina al cliente de la lista y manda a todos los clientes el evento ``all-connections'' para que actualicen ellos también la lista.

Por último, todos los clientes tienen el evento ``disconnect'', que permite que cuando el servidor se caiga por diversos motivos, cambie el mensaje de la página.

%INTRODUCIR FOTOS DE SALIDA CON VARIOS CLIENTES Y TODO ESO, JUNTO CON LAS DEL SERVIDOR

%El diagrama de interacciones es el siguiente:

%MOSTRAR SI ME VEO CAPAZ EL DIAGRAMA DE INTERACCION.
\subsection{mongo-test.js}
\subsubsection{Descripción}
Este ejemplo muestra al cliente todas las conexiones a la base de datos NoSQL, mostrando en un JSON el identificador, la dirección IP donde se encuentra el cliente, el puerto y la hora UTC a la que se ha realizado la conexión.

Cuando se refresca la página o se abre una pestaña nueva del navegador, aparecen nuevas líneas de conexiones a la base de datos.

%Mostrar una foto de la salida
\begin{figure}[H]
    \centering
    \includegraphics[width=0.7\textwidth]{images/mongo3.png}
    \caption{Tres conexiones distintas al servidor.}
\end{figure}

\subsubsection{Implementación}
La implementación del servidor HTTP es exactamente igual que en ejemplos anteriores, cuando se realiza una petición, el servidor envía la página HTML que se encarga de mostrar las conexiones.

En la parte de la base de datos, lo primero que se hace es conectarse, luego se crea una base de datos con nombre ``pruebaBaseDatos''.

\bigskip

A continuación, con Socket.io, se crean los eventos que tiene que escuchar y emitir: 

\begin{itemize}
    \item \textbf{my-address}: manda al cliente su dirección IP y puerto
    \item \textbf{poner:} Inserta los datos que el cliente manda en la base de datos.
    \item \textbf{obtener:} busca en la base de datos el registro dado por la dirección IP del cliente y se lo manda al mismo (el cliente).
    \item \textbf{poner:} Se encarga de insertar en la base de datos la información que ha enviado el cliente.
\end{itemize}

Para la parte del fichero HTML en el cliente, se realiza algo similar a ejemplos anteriores. Los eventos se encargan de insertar en la base de datos las nuevas conexiones y de actualizar la lista del cliente.

\newpage

Ahora bien, debido a que se usa el método ``createCollection'', solo se puede ejecutar el servidor una vez, en las siguientes ejecuciones da el siguiente error:

\begin{figure}[H]
    \centering
    \includegraphics[width=0.8\textwidth]{images/errormongo.png}
    \caption{Error de ejecución del servidor.}
\end{figure}

Esto es debido a que intenta crear de nuevo la colección y no lo consigue, haciendo que ``result'' sea ``null'' y se aborte la ejecución del servidor.

\subsection{Diferencias entre los ejemplos}
La principal diferencia entre los distintos ejemplos, es que en algunos se hace uso de un fichero HTML dedicado con JavaScript incrustado para el cliente (calculadora-web, connections y mongo-test), mientras que en otros tales como ``helloworld.js'' o ``calculadora.js'', el servidor solo incrusta el mensaje en una página básica sin hacer uso de un HTML dedicado.

\bigskip

Otra diferencia es que ``helloworld.js'', ``calculadora.js'' y ``calculadora-web.js'' solo hacen uso de Node.js, y algunos módulos del mismo. Mientras que en ``connections.js'' y ``mongo-test.js'' hacen uso de Socket.io adicionalmente.

\bigskip

Por último, ``mongo-test.js'' hace uso de un módulo (driver) para conectarse a una base de datos NoSQL MongoDB. Por tanto, este es el más complejo de todos.

\section{Segunda parte. Sistema domótico}
\subsection{Descripción}
En esta parte de la práctica se pide implementar un sistema de domótica básico que conste de dos sensores y dos actuadores. Los sensores reciben la información mediante un formulario, que en mi caso lo he hecho en una página HTML aparte. Además deben mostrar alertas cuando los valores superen unos límites y en caso de llegar al límite de temperatura o luminosidad, se debe cerrar la persiana automáticamente. Como extra, también he hecho que se accionen más actuadores en función de distintos valores de otros sensores.

También he realizado algunos extras más aparte, que explicaré más adelante para dotar al sistema de una mayor funcionalidad y autenticidad en la toma de decisiones sobre los actuadores.

\subsection{Implementación}
\subsubsection{Parte obligatoria}
Para esta parte lo que he hecho ha sido crear un servidor HTTP tal y como se realiza en los ejemplos proporcionados. Para la parte de la base de datos he realizado algo similar al ejemplo ``mongo-test.js'', he creado una base de datos denominada ``accionesSensores'' para almacenar todos los cambios que se realizan en los sensores a través del formulario.

\bigskip

Debido a que el ejemplo anteriormente mencionado usaba el método ``createCollection'', cuando el servidor se reiniciaba y la colección ya existía se producía un error y se paraba la ejecución del servidor. Tras revisar la documentación al final encontré el método ``collection'', muy similar al anterior, pero con la diferencia de que no tiene función fallback (se almacena la referencia en una variable) y que al existir ya una colección, simplemente obtiene la referencia y no intenta crearla de nuevo. En caso de no existir, sí crea la colección y devuelve la referencia a la misma, asegurando así que nunca falle por intentarla crearla de nuevo si ya existe.

\bigskip

Para la parte del agente, he realizado (solo en la parte básica, en la parte extra explicaré la otra función) una función, denominada ``comprobarSensor'', que se encarga de comprobar que los valores sensores estén dentro de unos umbrales, y en caso de no estarlos, crea una nueva alerta y la manda al cliente.

\bigskip

La interfaz web he intentado que sea más o menos consistente en términos de colores y de diseño. Tiene la siguiente forma:

%MOSTRAR IMAGEN DE LA PANTALLA PRINCIPAL
\begin{figure}[H]
    \centering
    \begin{minipage}[H]{0.49\textwidth}
        \centering
        \includegraphics[width=\textwidth]{images/pagina1.png}
    \end{minipage}
    \hfill
    \begin{minipage}[H]{0.49\textwidth}
        \centering
        \includegraphics[width=\textwidth]{images/pagina2.png}
    \end{minipage}
\end{figure}

\begin{figure}[H]
    \centering
    \includegraphics[width=0.7\textwidth]{images/pagina3.png}
\end{figure}

Como se puede ver, tiene otro sensor más y 3 actuadores más. Esto forma parte del extra, que explicaré más adelante. Además, si se pulsa en un actuador, este se activa o desactiva, reflejándose el cambio en los demás clientes conectados y en los que se van a conectar. Esto se consigue con el evento ``toggle-actuador-id'', que con el id del actuador cambia el estado del mismo en el servidor. Una vez hecho eso, mediante el evento ``cambio-actuador'' actualiza el recuadro de la interfaz web.

\begin{figure}[H]
    \centering
    \begin{minipage}[H]{0.49\textwidth}
        \centering
        \includegraphics[width=\textwidth]{images/humoff.png}
        \caption{Humidificador apagado.}
    \end{minipage}
    \hfill
    \begin{minipage}[H]{0.49\textwidth}
        \centering
        \includegraphics[width=\textwidth]{images/humon.png}
        \caption{Humidificador encendido}
    \end{minipage}
\end{figure}

También se puede observar que abajo tiene la opción de cambiar los valores de los sensores. Cuando se pulsa lleva a una página distinta con un formulario para cambiar el valor del sensor elegido en el desplegable:

%MOSTRAR FOTO DEL FORMULARIO
\begin{figure}[H]
    \centering
    \includegraphics[width=0.7\textwidth]{images/cambiarsensor.png}
    \caption{Botón para cambiar los valores de los sensores.}
\end{figure}

Un problema que tuve era que quería que al enviar el nuevo valor del sensor, se redireccionase a la página principal de nuevo, pero el emit que realizaba el socket no se llegaba a mandar al servidor, haciendo que los cambios no se produjesen. Al final opté por no redireccionar a ninguna página, esto hizo que ya funcionase, además así se permite modificar el valor de varios sensores a la vez. Cabe decir que cuando se envía el valor nuevo, este se guarda en la base de datos y se muestra en la página principal de todos los clientes que se encuentren en la misma.

%MOSTRAR FOTO DE UN CAMBIO, LUEGO DE LA PAGINA PRINCIPAL CON EL CAMBIO DEL SENSOR Y FINALMENTE OTRA CON EL CAMBIO REFLEJADO EN LA BASE DE DATOS.
\begin{figure}[H]
    \centering
    \includegraphics[width=0.7\textwidth]{images/form.png}
    \caption{Formulario para cambiar el valor del sensor.}
\end{figure}

\begin{figure}[H]
    \centering
    \includegraphics[width=\textwidth]{images/cambiotemp.png}
    \caption{Cambio del sensor de temperatura}
\end{figure}

\begin{figure}[H]
    \centering
    \includegraphics[width=0.7\textwidth]{images/cambiovalordb.png}
    \caption{Cambio reflejado en la página y en la base de datos.}
\end{figure}


Por último, se puede observar que arriba a la derecha hay una opción de alertas. Estas se activan (para la parte obligatoria, por ahora) cuando los sensores detectan un valor máximo o mínimo. Cuando aparece un número mayor que 0 significa que hay alguna alerta, y además, si se pulsa en el texto o en el recuadro, aparece un desplegable con todas las alertas más detalladas.

\begin{figure}[H]
    \centering
    \begin{minipage}[H]{0.49\textwidth}
        \centering
        \includegraphics[width=\textwidth]{images/alerta0.png}
        \caption{Ninguna alerta.}
    \end{minipage}
    \hfill
    \begin{minipage}[H]{0.49\textwidth}
        \centering
        \includegraphics[width=\textwidth]{images/alerta1.png}
        \caption{Hay una alerta}
    \end{minipage}
\end{figure}

\begin{figure}[H]
    \centering
    \includegraphics[width=0.7\textwidth]{images/alertamsg.png}
    \caption{Explicación de las alertas.}
\end{figure}

Ahora bien, he intentado generalizar lo máximo posible los sensores para que los eventos correspondientes puedan ser compartidos por todos y se simplifique la inserción de nuevos sensores en el futuro. Para ello, he creado un JSON con toda la información relativa de un sensor. Este a su vez se almacena en un array. El JSON tiene los siguientes campos:

\begin{itemize}
    \item \textbf{id: }Identificador del sensor.
    \item \textbf{name: }Nombre que se va a usar para las etiquetas HTML correspondientes a dicho sensor y poder direccionarlo en JavaScript.
    \item \textbf{sensorName: }Nombre del sensor que aparecerá en la pantalla principal.
    \item \textbf{unit: }Unidades usadas por el sensor.
    \item \textbf{maxWarningMsg: }Mensaje de alerta que se muestra cuando se llegue a ``highWarningValue''.
    \item \textbf{highWarningValue: }Valor superior a partir del cual ya se muestra una alerta.
    \item \textbf{redValue: }Valor límite que el usuario decide. Si se llega a este valor, el sistema tomará la decisión de accionar el actuador correspondiente.
    \item \textbf{maxValue: }Valor máximo que el sensor puede soportar.
    \item \textbf{minWarningMsg: }Mensaje de alerta que se muestra cuando se llegue a ``lowWarningValue''.
    \item \textbf{lowWarningValue: }Valor inferior a partir del cual ya se muestra una alerta.
    \item \textbf{blueValue: }Valor inferior límite que el usuario decide. Si se llega a este valor, el sistema tomará la decisión de accionar el actuador correspondiente.
    \item \textbf{minValue: }Valor mínimo que el sensor puede soportar.
    \item \textbf{imageDir: }Directorio donde se encuentra la imagen que se va a insertar.
    \item \textbf{currentValue: }Valor actual del sensor.
\end{itemize}

%Si da tiempo poner un dibujo de esto

He realizado lo mismo para los actuadores, que si bien, en la parte extra no se pueden crear más, me ha parecido más correcto que todos compartan el mismo evento y se haga de manera genérica.

\bigskip

Los actuadores tienen los siguientes campos:

\begin{itemize}
    \item \textbf{id: }Identificador del actuador.
    \item \textbf{name: }Nombre del actuador.
    \item \textbf{state: }Estado actual del actuador.
    \item \textbf{idName: }Identificador que se usa para direccionar en JavaScript.
    \item \textbf{img: }Ruta de la imagen que se va a usar.
\end{itemize}

\bigskip

Los eventos que se usan para la página principal son los siguientes:

\begin{itemize}
    \item \textbf{clientes: }Obtiene la lista de los clientes conectados en cada momento. Se actualiza cada vez que se conecta o desconecta algún cliente.
    \item \textbf{obtener-sensores: }Sirve para inicializar la pantalla con todos los sensores que hay en el sistema. También es accionado en el cliente cuando se añade o elimina un sensor.
    \item \textbf{cambio-sensor: }Se acciona en el cliente cuando se cambia el valor de un sensor para cambiarle el estilo si está en zona de alerta y actualizar el valor actual. Mientras que en el servidor se acciona cuando se envía el nuevo valor a través del formulario.
    \item \textbf{historial: }Actualiza la información del historial de cambios de los valores en los sensores. En el servidor se acciona cuando un cliente nuevo se conecta, mandándole todo el histórico de cambios en los sensores.
    \item \textbf{nuevo-registro: }Hace que el servidor envíe el nuevo registro que se ha producido al cambiar el valor de un sensor. Este evento extra permite una mayor eficiencia, ya que no se deben mandar todos los registros de nuevo y reconstruirlos. Además evita ralentizaciones de MongoDB.
    \item \textbf{alerta: }Actualiza las alertas actuales. Cabe destacar que este evento no se puede hacer más eficiente (es decir, mandando solo la información necesaria como pasaba con el historial de registros de la base de datos), ya que pueden eliminarse valores intermedios del array en el servidor.
    \item \textbf{obtener-actuadores: }Al igual que ``obtener-sensores'', sirve para obtener todos los actuadores disponibles en la página (cliente).
    \item \textbf{cambio-actuador: }Actualiza el estado del actuador cuando otro cliente lo activa/desactiva.
    \item \textbf{toggle-actuador-id: }Sirve para que el cliente alterne el estado del actuador con id pasada como parámetro. En el lado del servidor al recibirlo cambia el estado, lo retransmite a todos y comprueba si hay alguna alerta nueva.
\end{itemize}

No están explicados todos, ya que los demás son extra y los explicaré en la siguiente subsección.

\bigskip

Cuando un cliente se conecta, el servidor le manda todos los sensores, los actuadores, el historial, las alertas y los clientes conectados. Así se asegura que tenga la misma información que otros clientes.

%INTENTAR HACER UN DIAGRAMA PARA LA PAGINA PRINICIPAL
\bigskip

Para la parte del formulario de cambiar el valor de los sensores se hace uso del evento ``obtener-sensores'' para obtener los nombres de los sensores disponibles y mostrarlos en el desplegable y de ``obtener-sensor'', que se acciona cuando se envía la elección al servidor. Este a su vez emite mediante el evento ``cambio-sensor'' la información actualizada para que todos los clientes lo actualicen.

\bigskip

Por último, se pedía que cuando la temperatura y la luminosidad llegasen a un valor umbral, se cerrase la ventana. En mi caso lo que he hecho ha sido que cuando se llegue a cualquiera de los dos umbrales, se cierre la ventana, se accione el aire acondicionado y se apague el radiador.

%MOSTRAR UNA FOTO SOBRE ESTO CUANDO SE CAMBIA EL VALOR DE LUMINOSIDAD
\begin{figure}[H]
    \centering
    \includegraphics[width=0.7\textwidth]{images/muchaluz.png}
    \caption{Se ha encendido el aire acondicionado y se ha cerrado la ventana/persiana.}
\end{figure}

\newpage

\subsubsection{Parte extra}
Para la parte extra he realizado lo siguiente:

\begin{itemize}
    \item Otro sensor más y tres actuadores más.
    \item Opción de añadir y/o eliminar sensores.
    \item Más alertas de los valores de los sensores y de combinaciones de actuadores poco eficientes (por ejemplo, aire acondicionado y radiador encendido).
    \item Una simulación básica de cambio de temperatura y humedad con el tiempo, haciendo que el sistema sea autorregulable.
    \item Un sistema de chat grupal con inicio de sesión y registro necesario. Cuidando especialmente cuestiones de seguridad.
    \item Una pizarra digital distribuida en la que todos los clientes pueden dibujar a la vez y pueden a su vez ver los cambios de los demás en tiempo real.
\end{itemize}

He añadido un sensor de humedad. Mientras que de actuadores he añadido un radiador, un humidificador y un deshumidificador. Además, he añadido más lógica para las combinaciones de actuadores. Por ejemplo, si se enciende el radiador y el aire acondicionado, salta una alerta recomendando una acción. O si se enciende el humidificador y deshumidificador también salta otra alerta. Todo esto se comprueba con la función ``comprobarActuadores'', que se llama cada vez que se produce algún cambio en un actuador o en un sensor.

\bigskip

En total he realizado 4 posibles combinaciones ineficientes distintas.

%MOSTRAR FOTO DE UN ENTRELAZAMIENTO DE ACTUADORES
\begin{figure}[H]
    \centering
    \includegraphics[width=0.7\textwidth]{images/actuadoresmal.png}
    \caption{Muestra alertas sobre dispositivos encendidos.}
\end{figure}

\newpage

También he hecho un formulario con la posibilidad de añadir un nuevo sensor. Este formulario cuando se envía llama al evento ``add-sensor'' que envía el sensor al servidor, y este a su vez lo inserta en el array de sensores y mediante el evento ``obtener-sensores'' hace que se actualicen todos los clientes. 

%FOTO DE FORMULARIO PARA AÑADIR SENSOR
\begin{figure}[H]
    \centering
    \includegraphics[width=0.7\textwidth]{images/addsensor.png}
    \caption{Formulario para añadir sensor.}
\end{figure}

La opción de eliminar sensor hace exactamente lo mismo, pero con el evento ``delete-sensor'', que elimina del array ese sensor en el servidor y también llama a ``obtener-sensores'' para actualizar a los demás clientes.

%FOTO DE FORMULARIO PARA ELIMINAR SENSOR
\begin{figure}[H]
    \centering
    \includegraphics[width=0.7\textwidth]{images/deletesensorform.png}
    \caption{Formulario para eliminar un sensor.}
\end{figure}

\newpage

Además, estos nuevos sensores también pueden lanzar alertas cuando se superen los valores umbrales con mensajes personalizados introducidos en el formulario.

%FOTO DE LA ALERTA DE UN SENSOR DE PRUEBA
\begin{figure}[H]
    \centering
    \includegraphics[width=0.7\textwidth]{images/nuevosensoralerta.png}
    \caption{Alerta producida por el nuevo sensor.}
\end{figure}

Otra cosa que he añadido ha sido una simulación básica de humedad, temperatura y luminosidad (aunque esta última no me convence especialmente) que se puede accionar con un botón que se encuentra en la página principal, debajo de los actuadores. 

\begin{figure}[H]
    \centering
    \includegraphics[width=0.7\textwidth]{images/iniciarsim.png}
    \caption{Botón de simulación.}
\end{figure}


He supuesto que es un día de verano, por lo que si todos los actuadores están apagados va a subir la temperatura y bajar la humedad. Si la ventana esta bajada supongo total oscuridad, mientras que si esta subida, supongo un valor demasiado alto de luminosidad.

\bigskip

Además, cuando se acciona el aire acondicionado baja la temperatura, mientras que si está el radiador sube. Si está encendido el humidificador sube la humedad y si está encendido el deshumidificador baja. He supuesto que si los dos aparatos de cada pareja anteriormente mencionada están encendidos, se mantienen sus valores, excepto para el caso de la ventana que si esta subida aumenta un poco la temperatura también.

%FOTOS DE AUTORREGULACION
\begin{figure}[H]
    \centering
    \includegraphics[width=0.7\textwidth]{images/sim.png}
    \caption{Se ha activado el aire acondicionado porque la temperatura era excesiva.}
\end{figure}

En la parte superior izquierda de la página principal hay un texto que pone ``Opciones extra'' que si se pulsa sobre él aparece un menú.

%MOSTRAR FOTO DEL MENU
\begin{figure}[H]
    \centering
    \includegraphics[width=0.4\textwidth]{images/extra.png}
    \caption{Menú con opciones extra.}
\end{figure}

La parte del chat consiste en un grupo en el que todas las personas que estén registradas se pueden mandar y recibir mensajes a la vez.

\newpage

Antes que nada se presenta al usuario con una pantalla de login/registro. En caso de tener credenciales incorrectos en el login o tener un nombre existente en el registro, se muestra un error.

\begin{figure}[H]
    \centering
    \begin{minipage}[H]{0.49\textwidth}
        \centering
        \includegraphics[width=\textwidth]{images/login.png}
        \caption{Pantalla de login.}
    \end{minipage}
    \hfill
    \begin{minipage}[H]{0.49\textwidth}
        \centering
        \includegraphics[width=\textwidth]{images/errorlogin.png}
        \caption{Error de credenciales de login.}
    \end{minipage}
\end{figure}

Cuando el usuario inicia sesión es presentado al chat con todos los mensajes anteriores y además es capaz de poder enviar más.

%MOSTRAR FOTO DEL CHAT
\begin{figure}[H]
    \centering
    \includegraphics[width=0.6\textwidth]{images/chatmsg.png}
    \caption{Ejemplo de conversación.}
\end{figure}

Además, si el usuario sale del chat sin cerrar pestaña, puede seguir entrando sin iniciar sesión de nuevo. Esto se consigue utilizando ``sessionStorage'' de JavaScript.

También he puesto un censurador de palabrotas que pone con el símbolo `*' la palabra que detecte como mala.

%FOTO DEL CENSURADOR DE PALABROTAS
\begin{figure}[H]
    \centering
    \includegraphics[width=0.7\textwidth]{images/censura.png}
    \caption{Palabrota censurada.}
\end{figure}

Para implementar esto he creado dos colecciones en la base de datos: una para los usuarios registrados y otra para los mensajes enviados. Los eventos añadidos son los siguientes:

\begin{itemize}
    \item \textbf{recibir-todos-msgs: }Evento que obtiene todos los mensajes de la base de datos. Solo se acciona si el servidor ha comprobado que los credenciales son correctos.
    \item \textbf{recibir-msg: }Recibe el mensaje que ha enviado un usuario. El servidor además lo guarda en la base de datos y comprueba que lo ha enviado una persona que existe.
    \item \textbf{comprobar-cuenta: }Comprueba que los credenciales introducidos para iniciar sesión son válidos, en caso correcto, el cliente cambia la pantalla de login a la del chat y envía una petición a ``recibir-todos-msgs'' con sus credenciales para recibir los mensajes de chat.
    \item \textbf{crear-cuenta: }El servidor crea la cuenta almacenándola en la base de datos y devuelve true, en otro caso false. En el cliente al recibir el evento, realiza lo mismo que ``comprobar-cuenta''.
\end{itemize}    

Además, para la parte de inicio de sesión persistente, se tiene una parte ``main'' que se ejecuta si ve que tiene credenciales guardados en la sesión.

%MAS FOTOS DEL CHAT
\bigskip

La otra opción de ``Opciones extra'' es la de una pizarra digital del estilo de ``Google Jamboard''. Varios usuarios pueden dibujar a la vez y cada uno ve en su pantalla los dibujos del otro en tiempo real. Además tiene 3 colores, una goma de borrar, 3 grosores y una opción de borrar todo el lienzo. Todo esto lo ven también los demás.

\bigskip

Para la implementación de esto, he tenido que hacer un ``canvas'' con HTML y tener en cuenta que un dibujo es un conjunto de líneas, y que una línea está formada por dos puntos. Por tanto, hay que ser consciente de que el servidor debe almacenar un array de JSON con el grosor y el color del trazo, y con el punto de comienzo y de fin de la línea.

%FOTO DE LA PIZARRA EN DOS VENTANAS DISTINTAS
\begin{figure}[H]
    \centering
    \includegraphics[width=0.7\textwidth]{images/pizarra.png}
    \caption{Pizarra de los dos clientes. Se puede ver que tienen el mismo dibujo}
\end{figure}

\newpage

Para ello se han creado los siguientes eventos:

\begin{itemize}
    \item \textbf{update-pizarra: }Actualiza la pizarra con los nuevos trazos que se han introducido, ya sea por el propio usuario o por otro. Además se pinta del color que ha seleccionado el usuario que lo ha realizado.
    \item \textbf{limpiar-lienzo: }Limpia el lienzo de todos los clientes y elimina todos los datos del array del lienzo en el servidor.
\end{itemize}

Cuando un cliente nuevo se conecta, el servidor le envía todos los trazos mediante el evento ``update-pizarra''. A partir de ahí, ya el usuario puede pintar como los demás.

%OTRA FOTO MAS

Para finalizar la memoria, me gustaría mencionar que soy consciente de que el sistema no está libre de bugs, he intentado eliminar todos los que he podido. También no es lo más eficiente que me gustaría que fuera, repitiendo mucho código y seguramente con eventos que se podrían encapsular en otros.

Además, también quería hacer un bot de Twitter que escribiese tweets con todas las alertas en un momento; sin embargo pedí la cuenta de desarrollador demasiado tarde (sábado 21 de mayo) y no me ha dado tiempo a implementarlo.

\end{document}