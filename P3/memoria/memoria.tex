\documentclass{article}
\usepackage[utf8]{inputenc}
\usepackage{graphicx, graphics, float}
\usepackage[a4paper, total={6in, 10in}]{geometry}
\title{DSD Memoria Práctica 3}
\author{Andrés Merlo Trujillo}
\date{}
\begin{document}

\maketitle

\section{Ejemplos}
En esta primera parte se pide copiar los códigos de prueba que aparecen en el guión de prácticas y comentar como funcionan y qué hacen.

Lo primero que hay que hacer es tener instalado una versión de OpenJDK lo suficientemente antigua ya que si no al usar el comando \textit{javac *.java} saldrá un error similar a este:

%INSERTAR ERROR DE COMPILACIÓN

En mi caso tenía la versión 18 de JDK y he tenido que descargarme la versión 8 de la misma y configurarla para que sea la que sale por defecto.

Otro problema que he tenido esa la hora de usar el script, ya que dentro del mismo se lanzan secuencialmente el servidor y los clientes, esto produce que se lancen nada más que el servidor y que los clientes no se lancen.

%FOTO DE LO OCURRIDO
Para ello he separado la parte de la invocación de los clientes en otro script para poder lanzar ambos en dos terminales distintas.

\subsection{Ejemplo 1}
\subsubsection{Descripción}
Este ejemplo lo único que realiza es que el cliente posee un identificador pasado como argumento en la ejecución del cliente y este a su vez se lo pasa al servidor a través del propio método remoto.

En el lado del servidor al ejecutar el método hay dos opciones, si el identificador pasado acaba en 0 entonces duerme durante 5000ms (5 segundos), en otro caso simplemente indica que ha recibido la petición e indica el número de hebra.


\subsubsection{Implementación}
Para la implementación de este sistema primero es necesario realizar la interfaz del servidor que va a exportar los métodos a los clientes. En este caso sólo se exporta el método \textit{escribir\_mensaje}. También cabe destacar que debe heredar de \textit{Remote}.

En la parte del servidor se crea una clase que implementa los métodos de la interfaz ``Ejemplo\_I''.

El método imprime por la pantalla del servidor que ha recibido la petición del proceso. Además comprueba si el número de proceso es 0, en caso afirmativo se manda el servidor a dormir durante 5 segundos y luego, con independencia de que sea el proceso 0 ó no, se indica la el número del proceso.
\\
%HACE FALTA COMPROBAR QUE CREO QUE RMI ES MULTIHEBRA

\textbf{IMPORTANTE CAMBIAR TODO EL TEXTO QUE HE SUPUESTO QUE ES MULTIHEBRA Y NO LO ES}
\\\\
%COMPROBAR ESTO ULTIMO CON EL PDF
Luego en la parte del main del servidor activa el gestor de seguridad, que es exactamente igual para el cliente, crea una instancia de sí mismo (ya que el main se encuentra en la misma clase) y le indica al rmiregistry su dirección para que los clientes puedan encontrarlo...???

En la parte del cliente lo que hace es activar el gestor de seguridad, si no tenía uno ya, accede al \textit{rmiregistry} que se encuentra en la dirección pasada por consola, y finalmente obtiene a partir del registro la instancia para poder lanzar el método anteriormente mencionado.


\subsubsection{Funcionamiento}
El diagrama del sistema (sin incluir rmiregistry ni stubs, solo los propios objetos y clases) es el siguiente:
%INSERTAR DIAGRAMA QUE EXPLIQUE MAS O MENOS COMO FUNCIONA

Y ejemplos de ejecución son los siguientes:



\subsection{Ejemplo 2}

\subsection{Ejemplo 3}

\subsection{Diferencias entre los ejemplos}
\end{document}